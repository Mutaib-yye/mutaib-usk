\documentclass[12pt,a4paper]{article}
\usepackage[utf8]{inputenc}
\usepackage[bahasa]{babel}
\usepackage{graphicx}
\usepackage{geometry}
\usepackage{listings}
\usepackage{xcolor}
\usepackage{hyperref}
\usepackage{float}
\usepackage{tikz}
\usetikzlibrary{shapes.geometric, arrows, positioning}

\geometry{top=2.5cm, bottom=2.5cm, left=3cm, right=3cm}

\lstset{
    basicstyle=\ttfamily\small,
    breaklines=true,
    frame=single,
    backgroundcolor=\color{gray!10},
    commentstyle=\color{green!50!black},
    keywordstyle=\color{blue},
    stringstyle=\color{red},
    numbers=left,
    numberstyle=\tiny\color{gray},
    showstringspaces=false
}

\title{\textbf{LAPORAN UTS\\MANAJEMEN DAN PEMODELAN DATA\\WEB APLIKASI SISTEM INFORMASI PERKULIAHAN}}
\author{Nama: [NAMA ANDA]\\NIM: [NIM ANDA]\\Kelas: [KELAS ANDA]}
\date{\today}

\begin{document}

\maketitle
\newpage

\tableofcontents
\newpage

\section{Laporan Komputer Server dan Web Server}

\subsection{Spesifikasi Komputer/Server}

\subsubsection{Hardware}
\begin{itemize}
    \item \textbf{Processor}: [Sebutkan processor yang digunakan]
    \item \textbf{RAM}: [Sebutkan kapasitas RAM]
    \item \textbf{Storage}: [Sebutkan kapasitas penyimpanan]
    \item \textbf{Operating System}: [Sebutkan OS: Windows/Linux/MacOS]
\end{itemize}

\subsubsection{Software}
\begin{itemize}
    \item \textbf{Web Browser}: Google Chrome / Firefox / Safari
    \item \textbf{Runtime Environment}: Node.js v18+
    \item \textbf{Package Manager}: npm / bun
    \item \textbf{Development Server}: Vite Development Server
    \item \textbf{Database Server}: PostgreSQL 15+
\end{itemize}

\subsection{Konfigurasi Web Server}

Aplikasi ini menggunakan \textbf{Vite} sebagai development server dan build tool. Vite adalah modern frontend build tool yang menyediakan:

\begin{itemize}
    \item \textbf{Hot Module Replacement (HMR)}: Update instan tanpa refresh halaman
    \item \textbf{Port Default}: 5173 untuk development
    \item \textbf{Production Build}: Static files yang di-deploy ke hosting
\end{itemize}

\textbf{Konfigurasi Koneksi}:
\begin{lstlisting}[language=bash]
# Development Mode
npm run dev
# Akses via: http://localhost:5173

# Production Build
npm run build
# Deploy hasil build dari folder dist/
\end{lstlisting}

\subsection{Diagram Alur Komunikasi Client-Server}

\begin{figure}[H]
\centering
\begin{tikzpicture}[
    node distance=2cm,
    box/.style={rectangle, draw, fill=blue!20, text width=3cm, text centered, rounded corners, minimum height=1.2cm},
    arrow/.style={->, >=stealth, thick}
]
    \node[box] (client) {Client Browser};
    \node[box, right=of client] (vite) {Vite Server};
    \node[box, below=of vite] (api) {REST API};
    \node[box, right=of api] (db) {PostgreSQL Database};
    
    \draw[arrow] (client) -- node[above] {HTTP Request} (vite);
    \draw[arrow] (vite) -- node[right] {API Call} (api);
    \draw[arrow] (api) -- node[above] {SQL Query} (db);
    \draw[arrow] (db) -- node[below] {Data} (api);
    \draw[arrow] (api) -- node[left] {JSON Response} (vite);
    \draw[arrow] (vite) -- node[below] {HTML/JS/CSS} (client);
\end{tikzpicture}
\caption{Diagram Komunikasi Client-Server}
\end{figure}

\textbf{[Tambahkan screenshot konfigurasi server atau terminal saat menjalankan aplikasi]}

\begin{figure}[H]
\centering
\fbox{\parbox{0.9\textwidth}{\centering [SPACE FOR SCREENSHOT]\\\vspace{3cm}}}
\caption{Screenshot Server Running}
\end{figure}

\newpage
\section{Laporan Bahasa Pemrograman Web}

\subsection{Stack Teknologi}

Aplikasi ini dibangun menggunakan teknologi modern:

\begin{itemize}
    \item \textbf{Frontend Framework}: React 18 dengan TypeScript
    \item \textbf{Styling}: Tailwind CSS untuk responsive design
    \item \textbf{UI Components}: Radix UI dan shadcn/ui
    \item \textbf{State Management}: React Hooks (useState, useEffect, useContext)
    \item \textbf{Routing}: React Router v6
    \item \textbf{Backend}: PostgreSQL dengan REST API
    \item \textbf{Authentication}: JWT-based authentication
    \item \textbf{Form Validation}: Zod schema validation
\end{itemize}

\subsection{Peran Framework dalam Aplikasi}

\textbf{React} berperan sebagai:
\begin{enumerate}
    \item Component-based architecture untuk reusability
    \item Virtual DOM untuk performa optimal
    \item State management untuk data flow
    \item Hook system untuk logic reuse
\end{enumerate}

\textbf{TypeScript} menyediakan:
\begin{enumerate}
    \item Type safety untuk menghindari runtime errors
    \item Better IDE support dan autocomplete
    \item Interface definition untuk data structures
\end{enumerate}

\subsection{Contoh Kode CRUD Utama}

\subsubsection{Create - Input Nilai (DosenDashboard.tsx)}

\begin{lstlisting}[language=JavaScript]
const handleSaveNilai = async (e: React.FormEvent) => {
  e.preventDefault();
  
  // Validation
  const result = nilaiSchema.safeParse({ nilaiAngka });
  if (!result.success) {
    toast.error("Nilai harus antara 0-100");
    return;
  }
  
  // Check duplicate
  const { data: existing } = await supabase
    .from("nilai")
    .select("id")
    .eq("mahasiswa_id", selectedMahasiswa)
    .eq("mata_kuliah_id", selectedMataKuliah)
    .maybeSingle();
    
  if (existing) {
    toast.error("Nilai sudah ada");
    return;
  }
  
  // Insert data
  const { error } = await supabase
    .from("nilai")
    .insert([{
      mahasiswa_id: selectedMahasiswa,
      mata_kuliah_id: selectedMataKuliah,
      dosen_id: user?.id,
      nilai_angka: nilaiAngka,
      nilai_huruf: hitungNilaiHuruf(nilaiAngka)
    }]);
    
  if (error) {
    toast.error("Gagal menyimpan: " + error.message);
  } else {
    toast.success("Berhasil menyimpan nilai");
    fetchNilai();
  }
};
\end{lstlisting}

\subsubsection{Read - Fetch Data Nilai}

\begin{lstlisting}[language=JavaScript]
const fetchNilai = async () => {
  const { data, error } = await supabase
    .from("nilai")
    .select(`
      *,
      mahasiswa:profiles!nilai_mahasiswa_id_fkey(
        id, nama_lengkap, nim_nip
      ),
      mata_kuliah:mata_kuliah(
        id, kode_mk, nama_mata_kuliah, sks
      ),
      dosen:profiles!nilai_dosen_id_fkey(
        id, nama_lengkap
      )
    `)
    .eq("dosen_id", user.id);
    
  if (!error) {
    setNilai(data || []);
  }
};
\end{lstlisting}

\subsubsection{Update - Edit Nilai}

\begin{lstlisting}[language=JavaScript]
const handleUpdateNilai = async () => {
  const { error } = await supabase
    .from("nilai")
    .update({
      nilai_angka: nilaiAngka,
      nilai_huruf: hitungNilaiHuruf(nilaiAngka),
    })
    .eq("id", editingNilai.id);
    
  if (!error) {
    toast.success("Nilai berhasil diupdate");
    fetchNilai();
  }
};
\end{lstlisting}

\subsubsection{Delete - Hapus Data (AdminDashboard.tsx)}

\begin{lstlisting}[language=JavaScript]
const handleDeleteMataKuliah = async (id: string) => {
  const { error } = await supabase
    .from("mata_kuliah")
    .delete()
    .eq("id", id);
    
  if (error) {
    toast.error("Gagal menghapus");
  } else {
    toast.success("Berhasil dihapus");
    fetchMataKuliah();
  }
};
\end{lstlisting}

\subsection{Struktur Folder dan File Proyek}

\begin{lstlisting}[language=bash]
project-root/
|-- public/                    # Static assets
|   |-- robots.txt
|   |-- favicon.ico
|
|-- src/
|   |-- components/           # Reusable components
|   |   |-- ui/              # UI component library
|   |   |   |-- button.tsx
|   |   |   |-- card.tsx
|   |   |   |-- dialog.tsx
|   |   |   |-- table.tsx
|   |   |   |-- ... (30+ components)
|   |   |-- LanguageToggle.tsx
|   |   |-- ProtectedRoute.tsx
|   |   |-- Watermark.tsx
|   |
|   |-- contexts/            # React contexts
|   |   |-- LanguageContext.tsx
|   |
|   |-- hooks/               # Custom hooks
|   |   |-- useAuth.tsx
|   |   |-- use-toast.ts
|   |   |-- use-mobile.tsx
|   |
|   |-- pages/               # Main page components
|   |   |-- Index.tsx        # Landing page
|   |   |-- Auth.tsx         # Login/Register
|   |   |-- AdminDashboard.tsx
|   |   |-- DosenDashboard.tsx
|   |   |-- MahasiswaDashboard.tsx
|   |   |-- NotFound.tsx
|   |
|   |-- integrations/        # Backend integration
|   |   |-- supabase/
|   |       |-- client.ts    # Database client
|   |       |-- types.ts     # TypeScript types
|   |
|   |-- lib/                 # Utility functions
|   |   |-- utils.ts
|   |
|   |-- App.tsx              # Root component
|   |-- main.tsx             # Entry point
|   |-- index.css            # Global styles
|   |-- vite-env.d.ts        # Vite types
|
|-- supabase/
|   |-- migrations/          # Database migrations
|   |-- config.toml          # Supabase config
|
|-- index.html               # HTML template
|-- package.json             # Dependencies
|-- tsconfig.json            # TypeScript config
|-- tailwind.config.ts       # Tailwind config
|-- vite.config.ts           # Vite config
\end{lstlisting}

\textbf{[Tambahkan screenshot struktur folder dari IDE/Editor]}

\begin{figure}[H]
\centering
\fbox{\parbox{0.9\textwidth}{\centering [SPACE FOR SCREENSHOT]\\\vspace{4cm}}}
\caption{Screenshot Struktur Folder Proyek}
\end{figure}

\newpage
\section{Laporan Database Server (RDBMS)}

\subsection{Jenis Sistem Database}

Aplikasi ini menggunakan \textbf{PostgreSQL 15+} sebagai RDBMS dengan alasan:

\begin{itemize}
    \item Open-source dan gratis
    \item Mendukung ACID compliance
    \item Performa tinggi untuk query kompleks
    \item Mendukung JSON data type
    \item Row Level Security (RLS) untuk keamanan
    \item Rich indexing dan constraint support
\end{itemize}

\subsection{Struktur Tabel dalam Database}

Database terdiri dari 4 tabel utama:

\begin{enumerate}
    \item \textbf{profiles} - Menyimpan data profil pengguna
    \item \textbf{user\_roles} - Menyimpan role pengguna (admin/dosen/mahasiswa)
    \item \textbf{mata\_kuliah} - Menyimpan data mata kuliah
    \item \textbf{nilai} - Menyimpan data nilai mahasiswa
\end{enumerate}

\subsection{Detail Struktur Tabel}

\subsubsection{Tabel: profiles}

\begin{table}[H]
\centering
\begin{tabular}{|l|l|l|l|}
\hline
\textbf{Column} & \textbf{Type} & \textbf{Nullable} & \textbf{Default} \\
\hline
id & uuid & NO & - \\
email & text & NO & - \\
nama\_lengkap & text & NO & - \\
nim\_nip & text & YES & NULL \\
created\_at & timestamp & YES & now() \\
updated\_at & timestamp & YES & now() \\
\hline
\end{tabular}
\caption{Struktur Tabel profiles}
\end{table}

\textbf{Primary Key}: id (references auth.users.id)

\subsubsection{Tabel: user\_roles}

\begin{table}[H]
\centering
\begin{tabular}{|l|l|l|l|}
\hline
\textbf{Column} & \textbf{Type} & \textbf{Nullable} & \textbf{Default} \\
\hline
id & uuid & NO & gen\_random\_uuid() \\
user\_id & uuid & NO & - \\
role & app\_role (enum) & NO & - \\
created\_at & timestamp & YES & now() \\
\hline
\end{tabular}
\caption{Struktur Tabel user\_roles}
\end{table}

\textbf{Primary Key}: id\\
\textbf{Foreign Key}: user\_id → profiles.id\\
\textbf{Unique Constraint}: (user\_id, role)\\
\textbf{Enum Values}: 'admin', 'dosen', 'mahasiswa'

\subsubsection{Tabel: mata\_kuliah}

\begin{table}[H]
\centering
\begin{tabular}{|l|l|l|l|}
\hline
\textbf{Column} & \textbf{Type} & \textbf{Nullable} & \textbf{Default} \\
\hline
id & uuid & NO & gen\_random\_uuid() \\
kode\_mk & text & NO & - \\
nama\_mata\_kuliah & text & NO & - \\
sks & integer & NO & - \\
semester & integer & NO & - \\
created\_at & timestamp & YES & now() \\
updated\_at & timestamp & YES & now() \\
\hline
\end{tabular}
\caption{Struktur Tabel mata\_kuliah}
\end{table}

\textbf{Primary Key}: id\\
\textbf{Unique Constraint}: kode\_mk

\subsubsection{Tabel: nilai}

\begin{table}[H]
\centering
\begin{tabular}{|l|l|l|l|}
\hline
\textbf{Column} & \textbf{Type} & \textbf{Nullable} & \textbf{Default} \\
\hline
id & uuid & NO & gen\_random\_uuid() \\
mahasiswa\_id & uuid & NO & - \\
mata\_kuliah\_id & uuid & NO & - \\
dosen\_id & uuid & NO & - \\
nilai\_angka & numeric & YES & NULL \\
nilai\_huruf & text & YES & NULL \\
created\_at & timestamp & YES & now() \\
updated\_at & timestamp & YES & now() \\
\hline
\end{tabular}
\caption{Struktur Tabel nilai}
\end{table}

\textbf{Primary Key}: id\\
\textbf{Foreign Keys}:
\begin{itemize}
    \item mahasiswa\_id → profiles.id
    \item mata\_kuliah\_id → mata\_kuliah.id
    \item dosen\_id → profiles.id
\end{itemize}
\textbf{Unique Constraint}: (mahasiswa\_id, mata\_kuliah\_id)

\newpage
\section{Laporan Relational Database, SQL, dan Normalisasi}

\subsection{Entity Relationship Diagram (ERD)}

\begin{figure}[H]
\centering
\begin{tikzpicture}[
    entity/.style={rectangle, draw, fill=blue!20, text width=3cm, text centered, minimum height=1cm},
    relationship/.style={diamond, draw, fill=green!20, text width=2cm, text centered, aspect=2},
    attribute/.style={ellipse, draw, fill=yellow!20, text centered},
    node distance=3cm
]
    % Entities
    \node[entity] (profiles) {profiles};
    \node[entity, right=of profiles] (roles) {user\_roles};
    \node[entity, below=of profiles] (nilai) {nilai};
    \node[entity, right=of nilai] (mk) {mata\_kuliah};
    
    % Relationships
    \draw[->, thick] (profiles) -- node[above] {1:N} (roles);
    \draw[->, thick] (profiles) -- node[left] {1:N} (nilai);
    \draw[->, thick] (mk) -- node[above] {1:N} (nilai);
    
    % Labels
    \node[above=0.3cm of profiles] {User Profile};
    \node[above=0.3cm of roles] {User Roles};
    \node[below=0.3cm of nilai] {Student Grades};
    \node[above=0.3cm of mk] {Courses};
\end{tikzpicture}
\caption{Entity Relationship Diagram}
\end{figure}

\textbf{[Tambahkan diagram ERD yang lebih detail jika diperlukan]}

\begin{figure}[H]
\centering
\fbox{\parbox{0.9\textwidth}{\centering [SPACE FOR DETAILED ERD DIAGRAM]\\\vspace{5cm}}}
\caption{Detailed ERD with Attributes}
\end{figure}

\subsection{Proses Normalisasi}

\subsubsection{Bentuk Tidak Normal (Unnormalized Form)}

Data awal sebelum normalisasi mungkin terlihat seperti:

\begin{verbatim}
mahasiswa_data(nim, nama, email, mata_kuliah_list, 
               nilai_list, dosen_list, sks_list)
\end{verbatim}

Masalah: Data redundan, update anomaly, insertion anomaly

\subsubsection{First Normal Form (1NF)}

Eliminasi repeating groups dan pastikan atomicity:

\begin{itemize}
    \item Setiap kolom hanya berisi nilai atomic
    \item Tidak ada repeating groups
    \item Setiap tabel memiliki primary key
\end{itemize}

\textbf{Hasil}:
\begin{verbatim}
profiles(id, email, nama_lengkap, nim_nip)
nilai(id, mahasiswa_id, mata_kuliah_id, 
      dosen_id, nilai_angka, nilai_huruf)
\end{verbatim}

\subsubsection{Second Normal Form (2NF)}

Eliminasi partial dependencies:

\begin{itemize}
    \item Sudah dalam 1NF
    \item Semua non-key attributes fully dependent pada primary key
    \item Tidak ada partial dependency
\end{itemize}

\textbf{Hasil}: Pisahkan data mata kuliah ke tabel sendiri:
\begin{verbatim}
mata_kuliah(id, kode_mk, nama_mata_kuliah, sks, semester)
\end{verbatim}

\subsubsection{Third Normal Form (3NF)}

Eliminasi transitive dependencies:

\begin{itemize}
    \item Sudah dalam 2NF
    \item Tidak ada transitive dependencies
    \item Setiap non-key attribute hanya bergantung pada primary key
\end{itemize}

\textbf{Hasil}: Pisahkan role ke tabel terpisah:
\begin{verbatim}
user_roles(id, user_id, role)
\end{verbatim}

\textbf{Keuntungan Normalisasi}:
\begin{enumerate}
    \item Mengurangi redundansi data
    \item Meningkatkan integritas data
    \item Memudahkan maintenance
    \item Menghindari anomali insert, update, delete
\end{enumerate}

\subsection{Contoh Query SQL}

\subsubsection{SELECT dengan JOIN}

\begin{lstlisting}[language=SQL]
-- Mengambil data nilai lengkap dengan informasi mahasiswa,
-- mata kuliah, dan dosen
SELECT 
    n.id,
    n.nilai_angka,
    n.nilai_huruf,
    m.nama_lengkap AS nama_mahasiswa,
    m.nim_nip,
    mk.kode_mk,
    mk.nama_mata_kuliah,
    mk.sks,
    d.nama_lengkap AS nama_dosen
FROM nilai n
INNER JOIN profiles m ON n.mahasiswa_id = m.id
INNER JOIN mata_kuliah mk ON n.mata_kuliah_id = mk.id
INNER JOIN profiles d ON n.dosen_id = d.id
WHERE m.id = 'mahasiswa-uuid-here'
ORDER BY mk.semester, mk.nama_mata_kuliah;
\end{lstlisting}

\subsubsection{INSERT}

\begin{lstlisting}[language=SQL]
-- Insert nilai baru
INSERT INTO nilai (
    mahasiswa_id, 
    mata_kuliah_id, 
    dosen_id, 
    nilai_angka, 
    nilai_huruf
)
VALUES (
    'mahasiswa-uuid',
    'mata-kuliah-uuid',
    'dosen-uuid',
    85.5,
    'A'
);
\end{lstlisting}

\subsubsection{UPDATE}

\begin{lstlisting}[language=SQL]
-- Update nilai mahasiswa
UPDATE nilai
SET 
    nilai_angka = 90,
    nilai_huruf = 'A',
    updated_at = NOW()
WHERE 
    id = 'nilai-uuid'
    AND dosen_id = 'dosen-uuid';
\end{lstlisting}

\subsubsection{DELETE}

\begin{lstlisting}[language=SQL]
-- Hapus mata kuliah (hanya admin)
DELETE FROM mata_kuliah
WHERE id = 'mata-kuliah-uuid';
\end{lstlisting}

\subsubsection{Aggregate Functions}

\begin{lstlisting}[language=SQL]
-- Hitung IPK mahasiswa
SELECT 
    m.id,
    m.nama_lengkap,
    ROUND(
        SUM(
            CASE n.nilai_huruf
                WHEN 'A' THEN 4.0 * mk.sks
                WHEN 'B' THEN 3.0 * mk.sks
                WHEN 'C' THEN 2.0 * mk.sks
                WHEN 'D' THEN 1.0 * mk.sks
                ELSE 0
            END
        ) / NULLIF(SUM(mk.sks), 0),
        2
    ) AS ipk
FROM profiles m
LEFT JOIN nilai n ON m.id = n.mahasiswa_id
LEFT JOIN mata_kuliah mk ON n.mata_kuliah_id = mk.id
WHERE m.id = 'mahasiswa-uuid'
GROUP BY m.id, m.nama_lengkap;
\end{lstlisting}

\subsubsection{Complex JOIN dengan Filtering}

\begin{lstlisting}[language=SQL]
-- Ambil semua mahasiswa yang mengambil mata kuliah tertentu
-- beserta nilainya
SELECT 
    p.nama_lengkap,
    p.nim_nip,
    n.nilai_angka,
    n.nilai_huruf,
    mk.nama_mata_kuliah
FROM profiles p
INNER JOIN user_roles ur ON p.id = ur.user_id
INNER JOIN nilai n ON p.id = n.mahasiswa_id
INNER JOIN mata_kuliah mk ON n.mata_kuliah_id = mk.id
WHERE 
    ur.role = 'mahasiswa'
    AND mk.kode_mk = 'IF101'
ORDER BY n.nilai_angka DESC;
\end{lstlisting}

\newpage
\section{Laporan Implementasi Lokal Web Aplikasi}

\subsection{Struktur Folder Lokal Proyek}

Proyek disimpan di direktori lokal dengan struktur sebagai berikut:

\begin{lstlisting}[language=bash]
C:/Users/[Username]/projects/sistem-informasi-perkuliahan/
atau
/home/[username]/projects/sistem-informasi-perkuliahan/
\end{lstlisting}

\subsection{Langkah Menjalankan Aplikasi Lokal}

\subsubsection{Persiapan Environment}

\begin{enumerate}
    \item \textbf{Install Dependencies}
    \begin{lstlisting}[language=bash]
# Pastikan Node.js sudah terinstall (v18+)
node --version

# Install dependencies
npm install
# atau
bun install
    \end{lstlisting}
    
    \item \textbf{Konfigurasi Environment Variables}
    
    File \texttt{.env} sudah dikonfigurasi otomatis dengan:
    \begin{lstlisting}
VITE_SUPABASE_URL=your-project-url
VITE_SUPABASE_PUBLISHABLE_KEY=your-anon-key
    \end{lstlisting}
    
    \item \textbf{Setup Database}
    
    Database sudah dikonfigurasi dengan migrations di folder \texttt{supabase/migrations/}
\end{enumerate}

\subsubsection{Menjalankan Development Server}

\begin{lstlisting}[language=bash]
# Start development server
npm run dev

# Server akan berjalan di:
# http://localhost:5173
\end{lstlisting}

\subsection{Screenshot Halaman Aplikasi}

\subsubsection{Halaman Login}

\begin{figure}[H]
\centering
\fbox{\parbox{0.9\textwidth}{\centering [SPACE FOR SCREENSHOT - LOGIN PAGE]\\\vspace{5cm}}}
\caption{Halaman Login}
\end{figure}

\subsubsection{Dashboard Admin}

\begin{figure}[H]
\centering
\fbox{\parbox{0.9\textwidth}{\centering [SPACE FOR SCREENSHOT - ADMIN DASHBOARD]\\\vspace{5cm}}}
\caption{Dashboard Admin - Kelola Mata Kuliah dan User}
\end{figure}

\subsubsection{Dashboard Dosen}

\begin{figure}[H]
\centering
\fbox{\parbox{0.9\textwidth}{\centering [SPACE FOR SCREENSHOT - DOSEN DASHBOARD]\\\vspace{5cm}}}
\caption{Dashboard Dosen - Input Nilai}
\end{figure}

\subsubsection{Dashboard Mahasiswa}

\begin{figure}[H]
\centering
\fbox{\parbox{0.9\textwidth}{\centering [SPACE FOR SCREENSHOT - MAHASISWA DASHBOARD]\\\vspace{5cm}}}
\caption{Dashboard Mahasiswa - Lihat Transkrip Nilai}
\end{figure}

\subsection{Penjelasan Alur Fungsi Sistem}

\subsubsection{Alur Admin}

\begin{enumerate}
    \item Login dengan akun admin (rathermutaib333@gmail.com)
    \item Mengelola data mata kuliah:
    \begin{itemize}
        \item Tambah mata kuliah baru
        \item Edit data mata kuliah
        \item Hapus mata kuliah
    \end{itemize}
    \item Mengelola user:
    \begin{itemize}
        \item Lihat semua user
        \item Assign role (admin/dosen/mahasiswa)
        \item Update profile user
    \end{itemize}
\end{enumerate}

\subsubsection{Alur Dosen}

\begin{enumerate}
    \item Login dengan akun dosen
    \item Melihat daftar mata kuliah yang tersedia
    \item Melihat daftar mahasiswa
    \item Input nilai mahasiswa:
    \begin{itemize}
        \item Pilih mahasiswa
        \item Pilih mata kuliah
        \item Input nilai angka (0-100)
        \item Sistem otomatis menghitung nilai huruf
    \end{itemize}
    \item Edit nilai yang sudah diinput
    \item Lihat statistik nilai yang telah diinput
\end{enumerate}

\subsubsection{Alur Mahasiswa}

\begin{enumerate}
    \item Login dengan akun mahasiswa
    \item Melihat transkrip nilai:
    \begin{itemize}
        \item Daftar mata kuliah yang diambil
        \item Nilai angka dan huruf
        \item SKS per mata kuliah
        \item Nama dosen pengampu
    \end{itemize}
    \item Melihat statistik akademik:
    \begin{itemize}
        \item Total SKS diambil
        \item IPK (Indeks Prestasi Kumulatif)
        \item Jumlah mata kuliah
    \end{itemize}
\end{enumerate}

\subsection{Kendala Teknis dan Solusi}

\subsubsection{Kendala 1: Duplicate Entry}

\textbf{Problem}: Error "duplicate key value violates unique constraint" saat input nilai

\textbf{Solusi}: Implementasi checking sebelum insert:
\begin{lstlisting}[language=JavaScript]
const { data: existing } = await supabase
  .from("nilai")
  .select("id")
  .eq("mahasiswa_id", selectedMahasiswa)
  .eq("mata_kuliah_id", selectedMataKuliah)
  .maybeSingle();

if (existing) {
  toast.error("Nilai sudah ada");
  return;
}
\end{lstlisting}

\subsubsection{Kendala 2: Role-Based Access Control}

\textbf{Problem}: User bisa mengakses halaman yang bukan haknya

\textbf{Solusi}: Implementasi Protected Route dan Row Level Security (RLS)

\subsubsection{Kendala 3: Multilingual Support}

\textbf{Problem}: Aplikasi perlu mendukung dua bahasa (Bahasa Indonesia dan English)

\textbf{Solusi}: Implementasi Context API untuk language management

\newpage
\section{Hasil Web Aplikasi}

\subsection{Deployment Information}

Aplikasi dapat dijalankan dalam dua mode:

\subsubsection{Mode Localhost (Development)}

\begin{itemize}
    \item \textbf{URL}: http://localhost:5173
    \item \textbf{Server}: Vite Development Server
    \item \textbf{Hot Reload}: Enabled
    \item \textbf{Debug Mode}: Enabled
\end{itemize}

\subsubsection{Mode Production (Optional)}

\begin{itemize}
    \item Build production: \texttt{npm run build}
    \item Deploy ke hosting (Vercel, Netlify, dll)
    \item Optimized bundle
    \item Minified code
\end{itemize}

\subsection{Fitur Utama Aplikasi}

\subsubsection{Authentication System}

\begin{itemize}
    \item Login dengan email dan password
    \item Register akun baru
    \item Auto-assign role admin untuk email tertentu
    \item JWT-based session management
    \item Logout functionality
\end{itemize}

\subsubsection{Role-Based Dashboard}

\textbf{Admin Dashboard}:
\begin{itemize}
    \item CRUD mata kuliah
    \item Manage user roles
    \item View statistics
    \item Bilingual interface
\end{itemize}

\textbf{Dosen Dashboard}:
\begin{itemize}
    \item Input dan edit nilai mahasiswa
    \item View daftar mahasiswa
    \item View daftar mata kuliah
    \item Grade calculation (A-E)
\end{itemize}

\textbf{Mahasiswa Dashboard}:
\begin{itemize}
    \item View transkrip nilai
    \item Calculate IPK
    \item View course details
    \item View lecturer information
\end{itemize}

\subsection{Teknologi dan Keamanan}

\subsubsection{Security Features}

\begin{enumerate}
    \item \textbf{Row Level Security (RLS)}:
    \begin{itemize}
        \item Mahasiswa hanya bisa lihat nilai sendiri
        \item Dosen hanya bisa edit nilai yang diinput sendiri
        \item Admin bisa akses semua data
    \end{itemize}
    
    \item \textbf{Authentication}:
    \begin{itemize}
        \item Secure password hashing
        \item JWT token validation
        \item Protected routes
    \end{itemize}
    
    \item \textbf{Input Validation}:
    \begin{itemize}
        \item Zod schema validation
        \item Type-safe TypeScript
        \item Constraint checking
    \end{itemize}
\end{enumerate}

\subsubsection{Performance Optimization}

\begin{itemize}
    \item React component memoization
    \item Lazy loading
    \item Efficient database queries with proper indexes
    \item Tailwind CSS for optimized styling
\end{itemize}

\subsection{Testing dan Quality Assurance}

\begin{enumerate}
    \item \textbf{Functional Testing}:
    \begin{itemize}
        \item Login/logout functionality
        \item CRUD operations
        \item Role-based access
        \item Grade calculations
    \end{itemize}
    
    \item \textbf{UI/UX Testing}:
    \begin{itemize}
        \item Responsive design (mobile, tablet, desktop)
        \item Language toggle functionality
        \item Toast notifications
        \item Form validations
    \end{itemize}
\end{enumerate}

\subsection{Screenshot Aplikasi Berjalan}

\begin{figure}[H]
\centering
\fbox{\parbox{0.9\textwidth}{\centering [SPACE FOR SCREENSHOT - APP RUNNING]\\\vspace{4cm}}}
\caption{Screenshot Aplikasi Berjalan di Browser}
\end{figure}

\begin{figure}[H]
\centering
\fbox{\parbox{0.9\textwidth}{\centering [SPACE FOR SCREENSHOT - RESPONSIVE VIEW]\\\vspace{4cm}}}
\caption{Screenshot Responsive View (Mobile)}
\end{figure}

\newpage
\section{Kesimpulan}

Sistem Informasi Perkuliahan ini berhasil diimplementasikan dengan fitur-fitur lengkap:

\begin{enumerate}
    \item \textbf{Multi-role System}: Admin, Dosen, dan Mahasiswa dengan hak akses berbeda
    \item \textbf{CRUD Operations}: Create, Read, Update, Delete untuk semua entitas
    \item \textbf{Secure}: Row Level Security dan authentication system
    \item \textbf{User-friendly}: Modern UI dengan glassmorphism design
    \item \textbf{Multilingual}: Support Bahasa Indonesia dan English
    \item \textbf{Scalable}: Arsitektur yang dapat dikembangkan lebih lanjut
\end{enumerate}

\subsection{Future Improvements}

\begin{itemize}
    \item Export transkrip ke PDF
    \item Email notifications
    \item Advanced reporting dan analytics
    \item Attendance tracking
    \item Online assignment submission
\end{itemize}

\newpage
\appendix
\section{Lampiran}

\subsection{Database Schema SQL}

File lengkap database schema tersedia di \texttt{database-schema.sql}

\subsection{Environment Setup Guide}

Panduan lengkap setup environment:

\begin{lstlisting}[language=bash]
# 1. Install Node.js (v18 or higher)
# Download dari: https://nodejs.org

# 2. Clone atau extract project
cd sistem-informasi-perkuliahan

# 3. Install dependencies
npm install

# 4. Setup environment variables
# .env file sudah dikonfigurasi otomatis

# 5. Run development server
npm run dev

# 6. Build for production
npm run build
\end{lstlisting}

\subsection{Kode Sumber Penting}

\subsubsection{Authentication Hook (useAuth.tsx)}
\begin{lstlisting}[language=JavaScript]
// Custom hook untuk authentication
export const useAuth = () => {
  const navigate = useNavigate();
  const [user, setUser] = useState(null);
  const [loading, setLoading] = useState(true);

  useEffect(() => {
    supabase.auth.getSession().then(({ data: { session } }) => {
      setUser(session?.user ?? null);
      setLoading(false);
    });

    const { data: { subscription } } = supabase.auth.onAuthStateChange(
      (_event, session) => {
        setUser(session?.user ?? null);
      }
    );

    return () => subscription.unsubscribe();
  }, []);

  const signOut = async () => {
    await supabase.auth.signOut();
    navigate("/auth");
  };

  return { user, loading, signOut };
};
\end{lstlisting}

\subsubsection{Grade Calculation Logic}
\begin{lstlisting}[language=JavaScript]
const hitungNilaiHuruf = (angka: number): string => {
  if (angka >= 85) return "A";
  if (angka >= 70) return "B";
  if (angka >= 55) return "C";
  if (angka >= 40) return "D";
  return "E";
};

const hitungIPK = (nilaiList: any[]): number => {
  if (!nilaiList.length) return 0;
  
  let totalBobot = 0;
  let totalSKS = 0;
  
  nilaiList.forEach(n => {
    const sks = n.mata_kuliah.sks;
    let bobot = 0;
    
    switch(n.nilai_huruf) {
      case 'A': bobot = 4.0; break;
      case 'B': bobot = 3.0; break;
      case 'C': bobot = 2.0; break;
      case 'D': bobot = 1.0; break;
      default: bobot = 0;
    }
    
    totalBobot += bobot * sks;
    totalSKS += sks;
  });
  
  return totalSKS > 0 ? totalBobot / totalSKS : 0;
};
\end{lstlisting}

\end{document}